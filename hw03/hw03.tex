\documentclass{article}
\usepackage{eecstex}
\usepackage{circledsteps}

\title{EE 123 HW 03}
\author{Bryan Ngo}
\date{2022-02-07}

\begin{document}

\maketitle

\setcounter{section}{2}

\section{}

\subsection{}

\(y[n]\) is
\begin{itemize}
    \item \(y[n \geqslant L] = 0\) since an autocorrelation implies that the range of nonzero values is \([-L - 1, L - 1]\).
    \item Conjugate-symmetric since the frequency domain is purely real.
    \item Odd length since the multiplication of two length \(L\) signals in the frequency domain is the convolution, which creates a length \(2L - 1\).
\end{itemize}

\subsection{}

Due to the symmetry and convolution properties of the DFT,
\begin{equation}
    \bar{y}[n] = x[n] \, \Circled{N} \, x^\ast[-n]
\end{equation}
where we want
\begin{equation}
    y[n] = \bar{y}[m[n]] = x[n] \ast x^\ast[-n]
\end{equation}
Let \(N = 2L - 1\) so that we can capture the entire range of the circular autocorrelation.
Since circular convolution retains the same indices as the original signals, we now have a signal with range \([0, 2N - 2]\).
Thus, we must time shift by \(m[n] = n - L - 1\) to correct the range of the signal.

\newpage
\section{}

\begin{equation}
    \begin{array}[]{||c|c||}
        \hline
        k & X[k] \\
        \hline
        0 & 3 \\
        2 & 0.5 - 4.5j \\
        4 & 5 \\
        5 & 3.5 + 3.5j \\
        7 & -2.5 - 7j \\
        \hline
    \end{array}
\end{equation}

\subsection{}

Using the symmetry properties of the DFT, we can complete the DFT as shown:

\begin{equation}
    \begin{array}[]{||c|c||}
        \hline
        k & X[k] \\
        \hline
        0 & 3 \\
        1 & -2.5 + 7j \\
        2 & 0.5 - 4.5j \\
        3 & 3.5 - 3.5j \\
        4 & 5 \\
        5 & 3.5 + 3.5j \\
        6 & 0.5 + 4.5j \\
        7 & -2.5 - 7j \\
        \hline
    \end{array}
\end{equation}
meaning that
\begin{equation}
    x[0] = \frac{1}{8} \sum_{k = 0}^{N - 1} X[k] = \frac{11}{8}
\end{equation}

\subsection{}

By the circular convolution and time-shift theorems of the DFT,
\begin{equation}
    x[n] \, \Circled{8} \, \delta[n - 1] = e^{-j \frac{2\pi}{8} k} X[k]
\end{equation}

\subsection{}

Since \(W[k]\) is simply \(X[k]\) evaluated at the even terms, we can represent \(w[n]\) as
\begin{align}
    W[k] &= X[2k] = \sum_{n = 0}^7 x[n] W_8^{2kn} \\
    \overset{\mathcal{F}^{-1}}{\implies} w[n] &= \frac{1}{4} \sum_{k = 0}^3 W[k] W_4^{-kn} = \frac{1}{4} \sum_{k = 0}^3 X[2k] W_4^{-kn} \\
    &= \frac{1}{4} \sum_{k = 0}^7 \frac{1}{2} (1 + (-1)^k) X[k] W_8^{-kn} = \frac{1}{8} \sum_{k = 0}^7 (1 + e^{-j \pi k}) X[k] W_8^{-kn} \\
    &= x[n] + \sum_{k = 0}^7 e^{-j \pi k} X[k] W_8^{-kn} \\
    &= x[n] + \sum_{k = 0}^7 W_8^4 X[k] W_8^{-kn} = x[n] + \sum_{k = 0}^7 X[k] W_8^{-kn + 4} \\
    &= x[n] + x[(n - 4)_8]
\end{align}

\newpage
\section{Faster DFTs?}

Since \(f[n]\) and \(g[n]\) are both purely real, then \(j g[n]\) must be purely imaginary.
By the linearity of the DFT, \(H[k] = F[k] + j G[k]\).
We can then express \(F[k]\) and \(G[k]\) as
\begin{align}
    F[k] &= \Re\{H[k]\} = \frac{1}{2} (H[k] + H^\ast[k]) \\
    G[k] &= \Im\{H[k]\} = \frac{1}{2j} (H[k] - H^\ast[k])
\end{align}

\newpage
\section{Diagonalizing Circulant Matrices}

\begin{theorem}
    The DFT matrix diagonalizes all circulant matrices.
\end{theorem}
\begin{proof}
    Let \(\bm{H}\) be a circulant matrix
    \begin{equation}
        \bm{H} =
        \begin{bmatrix}
            c_1 & c_2 & \cdots & c_{N - 1} & c_N \\
            c_N & c_1 & \cdots & c_{N - 2} & c_{N - 1} \\
            \vdots & \vdots & \ddots & \vdots & \vdots \\
            c_2 & c_3 & \cdots & c_N & c_1
        \end{bmatrix} =
        \begin{bmatrix}
            c[n] \\
            c[(n - 1)_N] \\
            \vdots \\
            c[(n - (N - 1))_N]
        \end{bmatrix}
    \end{equation}
    and \(\bm{D}\) be the \(N \times N\) DFT matrix.
    Note that the DFT matrix satisfies the following relation: \(\bm{D}^{-1} = \frac{1}{N} \bm{D}^\ast\).
    Then,
    \begin{align}
        \bm{D}^{-1} \bm{H} \bm{D} &=
        \frac{1}{N} \begin{bmatrix}
            1 & 1 & \cdots & 1 \\
            1 & e^{j \frac{2\pi}{N}} & \cdots & e^{j \frac{2\pi}{N} (N - 1)} \\
            \vdots & \vdots & \ddots & \vdots \\
            1 & e^{j \frac{2\pi}{N} (N - 1)} & \cdots & e^{j \frac{2\pi}{n} (N - 1) (N - 1)}
        \end{bmatrix}
        \begin{bmatrix}
            c[n] \\
            c[(n - 1)_N] \\
            \vdots \\
            c[(n - (N - 1))_N]
        \end{bmatrix}
        \begin{bmatrix}
            1 & 1 & \cdots & 1 \\
            1 & e^{-j \frac{2\pi}{n}} & \cdots & e^{-j \frac{2\pi}{n} (n - 1)} \\
            \vdots & \vdots & \ddots & \vdots \\
            1 & e^{-j \frac{2\pi}{n} (n - 1)} & \cdots & e^{-j \frac{2\pi}{n} (n - 1) (n - 1)}
        \end{bmatrix} \\
        &=
        \frac{1}{N} \begin{bmatrix}
            C[k] \\
            W_N^{-1} C[k] \\
            \vdots \\
            W_N^{-(N - 1)} C[k]
        \end{bmatrix}
        \begin{bmatrix}
            1 & 1 & \cdots & 1 \\
            1 & e^{-j \frac{2\pi}{n}} & \cdots & e^{-j \frac{2\pi}{n} (n - 1)} \\
            \vdots & \vdots & \ddots & \vdots \\
            1 & e^{-j \frac{2\pi}{n} (n - 1)} & \cdots & e^{-j \frac{2\pi}{n} (n - 1) (n - 1)}
        \end{bmatrix} \\
        &=
        \frac{1}{N} \begin{bmatrix}
            N C[0] & 0 & \cdots & 0 \\
            0 & N C[0] & \cdots & 0 \\
            \vdots & \vdots & \ddots & \vdots \\
            0 & 0 & \cdots & N C[0]
        \end{bmatrix}
    \end{align}
    which is diagonal.
\end{proof}

% IDFT{X[k]} = 1/N (DFT(X*[k]))*

\end{document}
